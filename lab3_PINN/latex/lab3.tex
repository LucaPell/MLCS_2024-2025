\documentclass{article}
\usepackage[left = 20mm, right = 20mm]{geometry}
\linespread{1.2}
\usepackage{xcolor,eso-pic,soul}
\usepackage{graphicx}
\usepackage{enumitem}
\usepackage{pifont}
\usepackage{hyperref}
\usepackage{mathtools}
\graphicspath{{./figures}}
\definecolor{bluunipv}{HTML}{1E489F}
\newcommand{\step}[1]{\underline{\textbf{\large{#1}}} }  
\setulcolor{bluunipv}
\setul{0.5ex}{0.3ex}
\title{\textcolor{white}{\textbf{Machine Learning per il Calcolo Scientifico}\\ \small\textbf{Third laboratory exercises }}}

\date{}

% \AddToShipoutPictureBG{\color{bluunipv}%
% \AtPageUpperLeft{\rule[-20mm]{\paperwidth}{30cm}}%
% }

\begin{document}
    
    \definecolor{rossounipv}{HTML}{B2264A}
    \definecolor{bluunipv}{HTML}{1E489F}
    %\pagecolor{bluunipv}
    \AddToShipoutPicture*
    {%
      \AtPageUpperLeft
        {%
          \color{bluunipv}%
          \raisebox{-.1\paperheight}{\rule{\paperwidth}{.5\paperheight}}%

        }%
      %\color{white}%
     % \rule{\paperwidth}{.5\paperheight}%
    }
    \newgeometry{left = 20mm, right = 20mm, top = -6mm}
    \maketitle
    \begin{center}\step{General guide} \end{center}

    For each laboratory, an incomplete Python notebook will be provided with exercises (steps) that must be completed in the given order (some of the exercises will be needed in future laboratories). In step zero, all the Python packages that are needed in order to complete the notebook are listed. This PDF includes the text for the exercises and the expected outcomes for each step. While following the notebook is recommended, you are also welcome to attempt the exercises without using it.


    \begin{center}\step{Step Zero} \end{center}

    Here are the required Python (\url{https://www.python.org/}) packages for this laboratory:

    \begin{itemize}
      \item PyTorch (\url{https://pytorch.org/})
      \item Numpy (\url{https://numpy.org/})
      \item Matplotlib (\url{https://matplotlib.org/})
      \item[\textcolor{red}{\textbullet}] ipymlp (\url{https://matplotlib.org/ipympl/})
    \end{itemize}


    \begin{center}\step{Exercise one: Solve a simple ODE}\end{center}
    \begin{center}
      $\begin{cases}
          \displaystyle \frac{dy(t)}{dt} = - 2y(t),\ \ t\in (0,1)\\[15pt]
          y(0) = 1 
      \end{cases}$
    \end{center}


    a. Using DeepNet (DNN.py) define a FNN with 3 layer and 5 neuron, with tanh as activation function
    
    b. Using torch.rand define the training points (50), while torch.linspace for test (100) 

    c. Complete the functions eval\_loss\_ode and eval\_loss\_IC.

    d. Following the \href{https://pytorch.org/docs/stable/generated/torch.optim.Adam.html}{documentation} define the optimizer.

    e. Complete the training loop (2000 epochs).

    f. Visualize the results obtained by completing the plots written in the notebook

    \begin{center}\step{Exercise two: Non-linear ODE}\end{center}

    \begin{center}
      $\begin{cases}
        \displaystyle -\frac{d}{dx}\Big[(1+u^2)\frac{du}{dx}\Big] = -2\pi\cos(\pi x)^2\sin(\pi x) +  \pi^2 \sin(\pi x)\big(\sin(\pi x)^2 +1\big), \quad x \in (0,2). \\[15pt] 

        u(0)= u(2) = 0.
      \end{cases}$
    \end{center}

    a. Using DeepNet (DNN.py) define a FNN with 5 layer and 20 neuron, with tanh as activation function
    
    b. Using torch.rand define the training points (200), while torch.linspace for test (500) 

    c. Complete the functions eval\_loss\_ode and eval\_loss\_IC.

    d. Following the \href{https://pytorch.org/docs/stable/generated/torch.optim.Adam.html}{documentation} define the optimizer.

    e. Complete the training loop (5000 epochs).

    f. Visualize the results obtained by completing the plots written in the notebook
    \newpage
    \restoregeometry
  \begin{center}\step{Exercise three: Wave equation}\end{center}
  \begin{center}
    
    \begin{align*}
      \begin{cases}
        \displaystyle \frac{\partial^2 u}{\partial t^2}-2\frac{\partial^2 u}{\partial x^2}=0, & x\in [0,4], \ \ t\in[0,3],\\[10pt]
        \displaystyle u(x,0)= 0, &x\in[0,4], \\[10pt]
        \displaystyle \frac{\partial u}{\partial t}(x,0) =  \frac{\pi \sqrt2}{2}\sin \Big(\frac{\pi x}{2}\Big), &x\in[0,4], \\[10pt]
        \displaystyle u(0,t)=u(4,t)=0, &t\in[0,3].
      \end{cases}
    \end{align*}
  
  \end{center}

    a. Using DeepNet (DNN.py) define a FNN with 3 layer and 50 neuron, with tanh as activation function
    
    b. Using torch.rand.uniform define the training points (In order 1500,500,100,100).

    c. Complete the functions eval\_loss\_PDE and eval\_loss\_IC.

    d. Following the \href{https://pytorch.org/docs/stable/generated/torch.optim.Adam.html}{documentation} define the optimizer.

    e. Complete the training loop (2000 epochs).

    f. Visualize the results obtained by completing the plots written in the notebook

\end{document}
